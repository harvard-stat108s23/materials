% Options for packages loaded elsewhere
\PassOptionsToPackage{unicode}{hyperref}
\PassOptionsToPackage{hyphens}{url}
\PassOptionsToPackage{dvipsnames,svgnames,x11names}{xcolor}
%
\documentclass[
]{article}
\usepackage{amsmath,amssymb}
\usepackage{lmodern}
\usepackage{iftex}
\ifPDFTeX
  \usepackage[T1]{fontenc}
  \usepackage[utf8]{inputenc}
  \usepackage{textcomp} % provide euro and other symbols
\else % if luatex or xetex
  \usepackage{unicode-math}
  \defaultfontfeatures{Scale=MatchLowercase}
  \defaultfontfeatures[\rmfamily]{Ligatures=TeX,Scale=1}
\fi
% Use upquote if available, for straight quotes in verbatim environments
\IfFileExists{upquote.sty}{\usepackage{upquote}}{}
\IfFileExists{microtype.sty}{% use microtype if available
  \usepackage[]{microtype}
  \UseMicrotypeSet[protrusion]{basicmath} % disable protrusion for tt fonts
}{}
\makeatletter
\@ifundefined{KOMAClassName}{% if non-KOMA class
  \IfFileExists{parskip.sty}{%
    \usepackage{parskip}
  }{% else
    \setlength{\parindent}{0pt}
    \setlength{\parskip}{6pt plus 2pt minus 1pt}}
}{% if KOMA class
  \KOMAoptions{parskip=half}}
\makeatother
\usepackage{xcolor}
\usepackage[margin=1in]{geometry}
\usepackage{color}
\usepackage{fancyvrb}
\newcommand{\VerbBar}{|}
\newcommand{\VERB}{\Verb[commandchars=\\\{\}]}
\DefineVerbatimEnvironment{Highlighting}{Verbatim}{commandchars=\\\{\}}
% Add ',fontsize=\small' for more characters per line
\usepackage{framed}
\definecolor{shadecolor}{RGB}{248,248,248}
\newenvironment{Shaded}{\begin{snugshade}}{\end{snugshade}}
\newcommand{\AlertTok}[1]{\textcolor[rgb]{0.94,0.16,0.16}{#1}}
\newcommand{\AnnotationTok}[1]{\textcolor[rgb]{0.56,0.35,0.01}{\textbf{\textit{#1}}}}
\newcommand{\AttributeTok}[1]{\textcolor[rgb]{0.77,0.63,0.00}{#1}}
\newcommand{\BaseNTok}[1]{\textcolor[rgb]{0.00,0.00,0.81}{#1}}
\newcommand{\BuiltInTok}[1]{#1}
\newcommand{\CharTok}[1]{\textcolor[rgb]{0.31,0.60,0.02}{#1}}
\newcommand{\CommentTok}[1]{\textcolor[rgb]{0.56,0.35,0.01}{\textit{#1}}}
\newcommand{\CommentVarTok}[1]{\textcolor[rgb]{0.56,0.35,0.01}{\textbf{\textit{#1}}}}
\newcommand{\ConstantTok}[1]{\textcolor[rgb]{0.00,0.00,0.00}{#1}}
\newcommand{\ControlFlowTok}[1]{\textcolor[rgb]{0.13,0.29,0.53}{\textbf{#1}}}
\newcommand{\DataTypeTok}[1]{\textcolor[rgb]{0.13,0.29,0.53}{#1}}
\newcommand{\DecValTok}[1]{\textcolor[rgb]{0.00,0.00,0.81}{#1}}
\newcommand{\DocumentationTok}[1]{\textcolor[rgb]{0.56,0.35,0.01}{\textbf{\textit{#1}}}}
\newcommand{\ErrorTok}[1]{\textcolor[rgb]{0.64,0.00,0.00}{\textbf{#1}}}
\newcommand{\ExtensionTok}[1]{#1}
\newcommand{\FloatTok}[1]{\textcolor[rgb]{0.00,0.00,0.81}{#1}}
\newcommand{\FunctionTok}[1]{\textcolor[rgb]{0.00,0.00,0.00}{#1}}
\newcommand{\ImportTok}[1]{#1}
\newcommand{\InformationTok}[1]{\textcolor[rgb]{0.56,0.35,0.01}{\textbf{\textit{#1}}}}
\newcommand{\KeywordTok}[1]{\textcolor[rgb]{0.13,0.29,0.53}{\textbf{#1}}}
\newcommand{\NormalTok}[1]{#1}
\newcommand{\OperatorTok}[1]{\textcolor[rgb]{0.81,0.36,0.00}{\textbf{#1}}}
\newcommand{\OtherTok}[1]{\textcolor[rgb]{0.56,0.35,0.01}{#1}}
\newcommand{\PreprocessorTok}[1]{\textcolor[rgb]{0.56,0.35,0.01}{\textit{#1}}}
\newcommand{\RegionMarkerTok}[1]{#1}
\newcommand{\SpecialCharTok}[1]{\textcolor[rgb]{0.00,0.00,0.00}{#1}}
\newcommand{\SpecialStringTok}[1]{\textcolor[rgb]{0.31,0.60,0.02}{#1}}
\newcommand{\StringTok}[1]{\textcolor[rgb]{0.31,0.60,0.02}{#1}}
\newcommand{\VariableTok}[1]{\textcolor[rgb]{0.00,0.00,0.00}{#1}}
\newcommand{\VerbatimStringTok}[1]{\textcolor[rgb]{0.31,0.60,0.02}{#1}}
\newcommand{\WarningTok}[1]{\textcolor[rgb]{0.56,0.35,0.01}{\textbf{\textit{#1}}}}
\usepackage{graphicx}
\makeatletter
\def\maxwidth{\ifdim\Gin@nat@width>\linewidth\linewidth\else\Gin@nat@width\fi}
\def\maxheight{\ifdim\Gin@nat@height>\textheight\textheight\else\Gin@nat@height\fi}
\makeatother
% Scale images if necessary, so that they will not overflow the page
% margins by default, and it is still possible to overwrite the defaults
% using explicit options in \includegraphics[width, height, ...]{}
\setkeys{Gin}{width=\maxwidth,height=\maxheight,keepaspectratio}
% Set default figure placement to htbp
\makeatletter
\def\fps@figure{htbp}
\makeatother
\setlength{\emergencystretch}{3em} % prevent overfull lines
\providecommand{\tightlist}{%
  \setlength{\itemsep}{0pt}\setlength{\parskip}{0pt}}
\setcounter{secnumdepth}{-\maxdimen} % remove section numbering
\ifLuaTeX
  \usepackage{selnolig}  % disable illegal ligatures
\fi
\IfFileExists{bookmark.sty}{\usepackage{bookmark}}{\usepackage{hyperref}}
\IfFileExists{xurl.sty}{\usepackage{xurl}}{} % add URL line breaks if available
\urlstyle{same} % disable monospaced font for URLs
\hypersetup{
  pdftitle={Problem Set 5},
  pdfauthor={Insert Name},
  colorlinks=true,
  linkcolor={Maroon},
  filecolor={Maroon},
  citecolor={Blue},
  urlcolor={blue},
  pdfcreator={LaTeX via pandoc}}

\title{Problem Set 5}
\author{Insert Name}
\date{Stat 108, Week 8}

\begin{document}
\maketitle

\hypertarget{collaborators}{%
\subsubsection{Collaborators}\label{collaborators}}

I collaborated with\ldots{} (list names of collaborators here).

\begin{Shaded}
\begin{Highlighting}[]
\CommentTok{\# Put all necessary libraries here}
\FunctionTok{library}\NormalTok{(tidyverse)}
\end{Highlighting}
\end{Shaded}

\begin{verbatim}
## -- Attaching core tidyverse packages ------------------------ tidyverse 2.0.0 --
## v dplyr     1.1.0     v readr     2.1.4
## v forcats   1.0.0     v stringr   1.5.0
## v ggplot2   3.4.1     v tibble    3.1.8
## v lubridate 1.9.1     v tidyr     1.3.0
## v purrr     1.0.1     
## -- Conflicts ------------------------------------------ tidyverse_conflicts() --
## x dplyr::filter() masks stats::filter()
## x dplyr::lag()    masks stats::lag()
## i Use the ]8;;http://conflicted.r-lib.org/conflicted package]8;; to force all conflicts to become errors
\end{verbatim}

\hypertarget{due-wednesday-march-29th-at-1000pm}{%
\subsection{Due: Wednesday, March 29th at
10:00pm}\label{due-wednesday-march-29th-at-1000pm}}

\hypertarget{goals-of-this-problem-set}{%
\subsection{Goals of this problem set}\label{goals-of-this-problem-set}}

\begin{enumerate}
\def\labelenumi{\arabic{enumi}.}
\tightlist
\item
  Practice controlling the flow of your \texttt{R} code.
\item
  Practice creating and modifying functions.
\end{enumerate}

\hypertarget{note}{%
\subsection{Note}\label{note}}

In some of your chunks, you will be testing your functions and want to
determine how your functions behave when they error out. For those
chunks, include \texttt{error\ =\ TRUE} in the chunk options so that
your document still knits (but also displays the error message).

\hypertarget{problem-1}{%
\subsubsection{Problem 1}\label{problem-1}}

For this problem you will practice writing conditional statements. Make
sure to test out your conditionals using data from \texttt{trees}, the 8
trees around Portland's Woodstock Community Center.

\begin{Shaded}
\begin{Highlighting}[]
\FunctionTok{library}\NormalTok{(pdxTrees)}
\NormalTok{trees }\OtherTok{\textless{}{-}} \FunctionTok{get\_pdxTrees\_parks}\NormalTok{(}\AttributeTok{park =} \StringTok{"Woodstock Community Center"}\NormalTok{) }\SpecialCharTok{\%\textgreater{}\%}
  \FunctionTok{select}\NormalTok{(Common\_Name, DBH, Condition, Tree\_Height,}
\NormalTok{         Native, Edible)}
\end{Highlighting}
\end{Shaded}

\begin{enumerate}
\def\labelenumi{\alph{enumi}.}
\item
  Write a set of conditional(s) that satisfies the following
  requirements: If any of the trees in \texttt{trees} is a ``Sweetgum''
  and taller than 60 feet, print out ``Tall Sweetgums found''.
\item
  Write a set of conditional(s) that satisfies the following
  requirements: If any of the trees in \texttt{trees} do not have
  missing information for whether they are edible, print out ``Some
  values are not missing.''
\item
  Write a set of conditional(s) that satisfies the following
  requirements: If all of the Sweetgum trees in \texttt{trees} are
  taller than 80 feet, print out ``All Sweetgums found are very tall.''
  If only some of the Sweetgum trees in `trees are taller than 80 feet,
  print out ``Some Sweetgums found are very tall.''
\end{enumerate}

For Problems 2-4, feel free to keep using \texttt{trees} for testing but
you might also want a larger dataset for testing:

\begin{Shaded}
\begin{Highlighting}[]
\NormalTok{pdxTrees }\OtherTok{\textless{}{-}} \FunctionTok{get\_pdxTrees\_parks}\NormalTok{()}
\end{Highlighting}
\end{Shaded}

\hypertarget{problem-2}{%
\subsubsection{Problem 2}\label{problem-2}}

Figure out what the following code does and then turn it into a
function. For your new function, do the following:

\begin{itemize}
\tightlist
\item
  Test it.
\item
  Provide default values (when appropriate).
\item
  Use clear names for the function and arguments.
\item
  Make sure to appropriately handle missingness.

  \begin{itemize}
  \tightlist
  \item
    Hint: \texttt{length()} provides the number of entries (including
    \texttt{NA}s) of a vector.\\
  \end{itemize}
\item
  Check that any data inputs are the appropriate classes.

  \begin{itemize}
  \tightlist
  \item
    And, provide a helpful error message if they aren't.
  \end{itemize}
\item
  Generalize it by allowing the user to specify a confidence level.
\item
  Provide a warning message if the sample size is less than 30 using a
  conditional statement and \texttt{message()}.

  \begin{itemize}
  \tightlist
  \item
    In your message, report the issue and their sample size.
  \end{itemize}
\end{itemize}

\begin{Shaded}
\begin{Highlighting}[]
\NormalTok{thing1 }\OtherTok{\textless{}{-}} \FunctionTok{length}\NormalTok{(trees}\SpecialCharTok{$}\NormalTok{DBH)}
\NormalTok{thing2 }\OtherTok{\textless{}{-}} \FunctionTok{mean}\NormalTok{(trees}\SpecialCharTok{$}\NormalTok{DBH)}
\NormalTok{thing3 }\OtherTok{\textless{}{-}} \FunctionTok{sd}\NormalTok{(trees}\SpecialCharTok{$}\NormalTok{DBH)}\SpecialCharTok{/}\FunctionTok{sqrt}\NormalTok{(thing1)}
\NormalTok{thing4 }\OtherTok{\textless{}{-}} \FunctionTok{qt}\NormalTok{(}\AttributeTok{p =}\NormalTok{ .}\DecValTok{975}\NormalTok{, }\AttributeTok{df =}\NormalTok{ thing1 }\SpecialCharTok{{-}} \DecValTok{1}\NormalTok{)}
\NormalTok{thing5 }\OtherTok{\textless{}{-}}\NormalTok{ thing2 }\SpecialCharTok{{-}}\NormalTok{ thing4}\SpecialCharTok{*}\NormalTok{thing3}
\NormalTok{thing6 }\OtherTok{\textless{}{-}}\NormalTok{ thing2 }\SpecialCharTok{+}\NormalTok{ thing4}\SpecialCharTok{*}\NormalTok{thing3}
\end{Highlighting}
\end{Shaded}

\hypertarget{problem-3}{%
\subsubsection{Problem 3}\label{problem-3}}

While we (i.e.~Stat 108 students) all love the grammar of graphics, not
everyone does. For this problem, we are going to practice creating
wrapper functions for \texttt{ggplot2}.

Recall our discussion from class on tidy evaluation. If you want to
learn more, check out these pages:

\begin{itemize}
\tightlist
\item
  \href{https://ggplot2.tidyverse.org/articles/ggplot2-in-packages.html}{For
  using \texttt{ggplot2} functions in your own functions}
\item
  \href{https://dplyr.tidyverse.org/articles/programming.html}{For using
  \texttt{dplyr} functions in your own functions}
\end{itemize}

Here's our example of a wrapper for a histogram.

\begin{Shaded}
\begin{Highlighting}[]
\CommentTok{\# Minimal viable product working code}
\FunctionTok{ggplot}\NormalTok{(}\AttributeTok{data =}\NormalTok{ pdxTrees, }\AttributeTok{mapping =} \FunctionTok{aes}\NormalTok{(}\AttributeTok{x =}\NormalTok{ DBH)) }\SpecialCharTok{+}
  \FunctionTok{geom\_histogram}\NormalTok{()}
\end{Highlighting}
\end{Shaded}

\begin{verbatim}
## `stat_bin()` using `bins = 30`. Pick better value with `binwidth`.
\end{verbatim}

\includegraphics{pset05_files/figure-latex/unnamed-chunk-8-1.pdf}

\begin{Shaded}
\begin{Highlighting}[]
\CommentTok{\# Function}
\NormalTok{histo }\OtherTok{\textless{}{-}} \ControlFlowTok{function}\NormalTok{(data, x, ...)\{}
  \CommentTok{\#Determine if x is numeric}
\NormalTok{  x\_class }\OtherTok{\textless{}{-}}\NormalTok{ data }\SpecialCharTok{\%\textgreater{}\%}
    \FunctionTok{pull}\NormalTok{(\{\{ x \}\}) }\SpecialCharTok{\%\textgreater{}\%}
    \FunctionTok{is.numeric}\NormalTok{()}
  \FunctionTok{stopifnot}\NormalTok{(x\_class)}
  
  \FunctionTok{ggplot}\NormalTok{(}\AttributeTok{data =}\NormalTok{ data, }\AttributeTok{mapping =} \FunctionTok{aes}\NormalTok{(}\AttributeTok{x =}\NormalTok{ \{\{ x \}\})) }\SpecialCharTok{+}
    \FunctionTok{geom\_histogram}\NormalTok{()}
\NormalTok{\}}

\CommentTok{\# Test it}
\FunctionTok{histo}\NormalTok{(pdxTrees, DBH)}
\end{Highlighting}
\end{Shaded}

\begin{verbatim}
## `stat_bin()` using `bins = 30`. Pick better value with `binwidth`.
\end{verbatim}

\includegraphics{pset05_files/figure-latex/unnamed-chunk-8-2.pdf}

\begin{enumerate}
\def\labelenumi{\alph{enumi}.}
\tightlist
\item
  Edit \texttt{histo()} so that the user can set
\end{enumerate}

\begin{itemize}
\tightlist
\item
  The number of bins
\item
  The fill color for the bars
\item
  The color outlining the bars
\end{itemize}

\begin{enumerate}
\def\labelenumi{\alph{enumi}.}
\setcounter{enumi}{1}
\item
  Write code to create a basic scatterplot with \texttt{ggplot2}. Then
  write and test a function to create a basic scatterplot.
\item
  Modify your scatterplot function to allow the user to:
\end{enumerate}

\begin{itemize}
\tightlist
\item
  Color the points by another variable.
\item
  Set the transparency.

  \begin{itemize}
  \tightlist
  \item
    And include a check that the transparency input is within the
    appropriate range.
  \end{itemize}
\end{itemize}

Also check the inputs.

\begin{enumerate}
\def\labelenumi{\alph{enumi}.}
\setcounter{enumi}{3}
\tightlist
\item
  Write and test a function for your favorite \texttt{ggplot2} graph.
  Make sure to give the user at least 3 optional inputs that they can
  change to customize the plot.
\end{enumerate}

\hypertarget{problem-4}{%
\subsubsection{Problem 4:}\label{problem-4}}

Who thinks it is a bit clunky to get conditional proportions using
\texttt{dplyr}? Let's practice writing functions for common data
wrangling operations.

\begin{enumerate}
\def\labelenumi{\alph{enumi}.}
\tightlist
\item
  Take the following code and turn it into an R function to create a
  conditional proportions table. Similar to \texttt{ggplot2}, you will
  need to handle the tidy evaluation. And, make sure to test your
  function!
\end{enumerate}

\begin{Shaded}
\begin{Highlighting}[]
\NormalTok{pdxTrees }\SpecialCharTok{\%\textgreater{}\%}
  \FunctionTok{count}\NormalTok{(Native, Condition) }\SpecialCharTok{\%\textgreater{}\%}
  \FunctionTok{group\_by}\NormalTok{(Native) }\SpecialCharTok{\%\textgreater{}\%}
  \FunctionTok{mutate}\NormalTok{(}\AttributeTok{prop =}\NormalTok{ n}\SpecialCharTok{/}\FunctionTok{sum}\NormalTok{(n)) }\SpecialCharTok{\%\textgreater{}\%}
  \FunctionTok{ungroup}\NormalTok{()}
\end{Highlighting}
\end{Shaded}

\begin{verbatim}
## # A tibble: 10 x 4
##    Native Condition     n    prop
##    <chr>  <chr>     <int>   <dbl>
##  1 No     Fair      12284 0.865  
##  2 No     Good       1043 0.0734 
##  3 No     Poor        875 0.0616 
##  4 Yes    Fair       9877 0.904  
##  5 Yes    Good        600 0.0549 
##  6 Yes    Poor        454 0.0415 
##  7 <NA>   Dead        264 0.658  
##  8 <NA>   Fair        118 0.294  
##  9 <NA>   Good          3 0.00748
## 10 <NA>   Poor         16 0.0399
\end{verbatim}

\begin{enumerate}
\def\labelenumi{\alph{enumi}.}
\setcounter{enumi}{1}
\tightlist
\item
  Write a function to compute the mean, median, sd, min, max, sample
  size, and number of missing values of a quantitative variable by the
  categories of another variable. Make sure the output is a data frame
  (or tibble).
\end{enumerate}

\end{document}
